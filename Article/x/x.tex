% Options for packages loaded elsewhere
\PassOptionsToPackage{unicode}{hyperref}

% Set document class options
\documentclass[webpdf,large,contemporary,namedate]{oup-authoring-template}

% one column


%\usepackage{showframe}

% line numbers

% use upquote if available, for straight quotes in verbatim environments
\IfFileExists{upquote.sty}{\usepackage{upquote}}{}

% From Pandoc template for its feature
\usepackage{xcolor}
\usepackage{hyperref}

\hypersetup{
  pdftitle={Tolerance of hybrid hosts against infections, TAC Meeting December 2022},
  pdfkeywords={hybrids, parasites},
  breaklinks=true,
  bookmarks=true,
  hidelinks,
  pdfcreator={LaTeX via pandoc}}


% Pandoc syntax highlighting
\usepackage{color}
\usepackage{fancyvrb}
\newcommand{\VerbBar}{|}
\newcommand{\VERB}{\Verb[commandchars=\\\{\}]}
\DefineVerbatimEnvironment{Highlighting}{Verbatim}{commandchars=\\\{\}}
% Add ',fontsize=\small' for more characters per line
\usepackage{framed}
\definecolor{shadecolor}{RGB}{248,248,248}
\newenvironment{Shaded}{\begin{snugshade}}{\end{snugshade}}
\newcommand{\AlertTok}[1]{\textcolor[rgb]{0.94,0.16,0.16}{#1}}
\newcommand{\AnnotationTok}[1]{\textcolor[rgb]{0.56,0.35,0.01}{\textbf{\textit{#1}}}}
\newcommand{\AttributeTok}[1]{\textcolor[rgb]{0.77,0.63,0.00}{#1}}
\newcommand{\BaseNTok}[1]{\textcolor[rgb]{0.00,0.00,0.81}{#1}}
\newcommand{\BuiltInTok}[1]{#1}
\newcommand{\CharTok}[1]{\textcolor[rgb]{0.31,0.60,0.02}{#1}}
\newcommand{\CommentTok}[1]{\textcolor[rgb]{0.56,0.35,0.01}{\textit{#1}}}
\newcommand{\CommentVarTok}[1]{\textcolor[rgb]{0.56,0.35,0.01}{\textbf{\textit{#1}}}}
\newcommand{\ConstantTok}[1]{\textcolor[rgb]{0.00,0.00,0.00}{#1}}
\newcommand{\ControlFlowTok}[1]{\textcolor[rgb]{0.13,0.29,0.53}{\textbf{#1}}}
\newcommand{\DataTypeTok}[1]{\textcolor[rgb]{0.13,0.29,0.53}{#1}}
\newcommand{\DecValTok}[1]{\textcolor[rgb]{0.00,0.00,0.81}{#1}}
\newcommand{\DocumentationTok}[1]{\textcolor[rgb]{0.56,0.35,0.01}{\textbf{\textit{#1}}}}
\newcommand{\ErrorTok}[1]{\textcolor[rgb]{0.64,0.00,0.00}{\textbf{#1}}}
\newcommand{\ExtensionTok}[1]{#1}
\newcommand{\FloatTok}[1]{\textcolor[rgb]{0.00,0.00,0.81}{#1}}
\newcommand{\FunctionTok}[1]{\textcolor[rgb]{0.00,0.00,0.00}{#1}}
\newcommand{\ImportTok}[1]{#1}
\newcommand{\InformationTok}[1]{\textcolor[rgb]{0.56,0.35,0.01}{\textbf{\textit{#1}}}}
\newcommand{\KeywordTok}[1]{\textcolor[rgb]{0.13,0.29,0.53}{\textbf{#1}}}
\newcommand{\NormalTok}[1]{#1}
\newcommand{\OperatorTok}[1]{\textcolor[rgb]{0.81,0.36,0.00}{\textbf{#1}}}
\newcommand{\OtherTok}[1]{\textcolor[rgb]{0.56,0.35,0.01}{#1}}
\newcommand{\PreprocessorTok}[1]{\textcolor[rgb]{0.56,0.35,0.01}{\textit{#1}}}
\newcommand{\RegionMarkerTok}[1]{#1}
\newcommand{\SpecialCharTok}[1]{\textcolor[rgb]{0.00,0.00,0.00}{#1}}
\newcommand{\SpecialStringTok}[1]{\textcolor[rgb]{0.31,0.60,0.02}{#1}}
\newcommand{\StringTok}[1]{\textcolor[rgb]{0.31,0.60,0.02}{#1}}
\newcommand{\VariableTok}[1]{\textcolor[rgb]{0.00,0.00,0.00}{#1}}
\newcommand{\VerbatimStringTok}[1]{\textcolor[rgb]{0.31,0.60,0.02}{#1}}
\newcommand{\WarningTok}[1]{\textcolor[rgb]{0.56,0.35,0.01}{\textbf{\textit{#1}}}}

% tightlist command for lists without linebreak
\providecommand{\tightlist}{%
  \setlength{\itemsep}{0pt}\setlength{\parskip}{0pt}}




% Counters for addresses and footnotes
\newcounter{correspcnt} % For author footnotes
\renewcommand*{\thecorrespcnt}{\fnsymbol{correspcnt}}
\newcounter{addrcnt} % For author addresses

% Macros for dealing with affiliations, footnotes, etc.
\makeatletter

\def\MyNewLabel#1#2#3{\expandafter\gdef\csname #1@#2\endcsname{#3}}

\def\MyRef#1#2{\@ifundefined{#1@#2}{???}{\csname #1@#2\endcsname}}

\newcommand*\ifcounter[1]{%
  \ifcsname c@#1\endcsname
    \expandafter\@firstoftwo
  \else
    \expandafter\@secondoftwo
  \fi
}

\newcommand*\addrlblbycode[1]{%
  \ifcounter{ADDRLBL@#1}
    {}
    {\refstepcounter{addrcnt}\newcounter{ADDRLBL@#1}\setcounter{ADDRLBL@#1}{\value{addrcnt}}}%
    \arabic{ADDRLBL@#1}%
}

\newcommand*\addrbycode[1]{%
  \ifcounter{ADDR@#1}
    {}
    {\newcounter{ADDR@#1}%
     \address[\addrlblbycode{#1}]{\MyRef{ADDRTXT}{#1}}}%
}

\newcommand*\corresplblbycode[1]{%
  \ifcounter{CORRESPLBL@#1}
    {}
    {\refstepcounter{correspcnt}\newcounter{CORRESPLBL@#1}\setcounter{CORRESPLBL@#1}{\value{correspcnt}}}%
    \fnsymbol{CORRESPLBL@#1}%
}

\newcommand*\correspbycode[1]{%
  \ifcounter{CORRESP@#1}
    {}
    {\newcounter{CORRESP@#1}%
     \corresp[\corresplblbycode{#1}]{\MyRef{CORRESPTXT}{#1}}}%
}

\makeatother

% Add missing \city command mentioned in documentation but absent from cls
\providecommand\city[1]{#1}

% Create labels for Addresses if the are given in Elsevier format
   \MyNewLabel{ADDRTXT}{ABC}{%
  %
  \orgdiv{Department of Molecular Parasitology, Institute for
Biology}, %
  \orgname{Humboldt University Berlin (HU)}, %
  \orgaddress{%
   %
  \street{Philippstraße 13}, %
  %
  \city{Berlin}, %
  \state{10115}, %
  \country{Germany}%
  }%
   %
 %
 }
   \MyNewLabel{ADDRTXT}{DEF}{%
  %
  \orgdiv{Research Group Ecology and Evolution of Molecular
Parasite-Host Interactions}, %
  \orgname{Leibniz-Institut for Zoo and Wildlife Research (IZW)}, %
  \orgaddress{%
   %
  \street{Alfred-Kowalke-Straße 17}, %
  %
  \city{Berlin}, %
  \state{10315}, %
  \country{Germany}%
  }%
   %
 %
 }

% Create labels for Footnotes if they are given in Elsevier format
\MyNewLabel{CORRESPTXT}{1}{}
\MyNewLabel{CORRESPTXT}{2}{Current email address:
\href{mailto:fay.webster@hu-berlin.de}{fay.webster@hu-berlin.de)})}

% Pandoc header-include feature
\theoremstyle{thmstyleone}
\newtheorem{theorem}{Theorem}
\newtheorem{proposition}[theorem]{Proposition}
\theoremstyle{thmstyletwo}
\newtheorem{example}{Example}
\newtheorem{remark}{Remark}
\theoremstyle{thmstylethree}
\newtheorem{definition}{Definition}
% Pandoc header-include feature
\usepackage{booktabs}

\begin{document}

\journaltitle{Journal Title Here}
\DOI{DOI HERE}
\copyrightyear{YYYY}
\pubyear{YYYY}
\access{Advance Access Publication Date: Day Month Year}
\appnotes{Paper}

\firstpage{1}



\title[]{Tolerance of hybrid hosts against infections, TAC Meeting
December 2022}

\newcounter{thisauthcorresp} % For storage if author is corresponding author
\newcounter{thisauththanks} % For storage if author has thanks



\author[%
\addrlblbycode{ABC},\addrlblbycode{DEF}%
,\refstepcounter{correspcnt}\setcounter{thisauthcorresp}{\value{correspcnt}}\fnsymbol{thisauthcorresp}%
%
%
]{Fay Webster}

\addrbycode{ABC}
\addrbycode{DEF}

\corresp[\fnsymbol{thisauthcorresp}]{Corresponding author. \href{mailto:fay.webster@hu-berlin.de}{\nolinkurl{fay.webster@hu-berlin.de}}}




\author[%
\addrlblbycode{ABC},\addrlblbycode{DEF}%
%
%
,\corresplblbycode{2}%
]{Lubomír Bednář}

\addrbycode{ABC}
\addrbycode{DEF}




\correspbycode{2}


\author[%
\addrlblbycode{ABC},\addrlblbycode{DEF}%
%
%
,\corresplblbycode{1}%
]{Emanuel Heitlinger}

\addrbycode{ABC}
\addrbycode{DEF}




\correspbycode{1}


% Add author mark
\authormark{Fay Webster et al.}

\received{Date}{0}{Year}
\revised{Date}{0}{Year}
\accepted{Date}{0}{Year}

%\editor{Associate Editor: Name}

\abstract{
Parasites in hybrid zones can give insight into species barriers, as
they are modulating the fitness of hybrid hosts. Recent findings have
demonstrated lower infection intensities with parasites in hybrids in
the European House Mouse Hybrid zone (HMHZ), indicating higher disease
resistance. However, tolerance has not yet been addressed in depth, as
it is impractical to measure in wild populations. In an attempt to
predict and evaluate the health impact of parasite infections and
extrapolate tolerance in the HMHZ, we use a machine learning method. A
random forest model was trained on immune parameters measured in
experimental lab infections with Eimeria and then applied to data
obtained from field sampling. Our predictions revealed that these
infections are more detrimental to hybrid male mice. This approach
represents an initial step in assessing tolerance in field studies.}

\keywords{hybrids; parasites}


\maketitle


\hypertarget{introduction}{%
\section{Introduction}\label{introduction}}

\hypertarget{introduce-the-topic}{%
\subsection{Introduce the topic}\label{introduce-the-topic}}

\hypertarget{methods}{%
\section{Methods}\label{methods}}

\hypertarget{statistical-analysis}{%
\subsection{Statistical Analysis}\label{statistical-analysis}}

\hypertarget{imputation-of-missing-data}{%
\subsubsection{Imputation of missing
data}\label{imputation-of-missing-data}}

To make the most of our data collection, we aimed to resolve
missingness. Missing data were imputed using multiple imputations by
chained equations. We used the package MICE in R \citet{van2011mice},
with five imputed data sets and five iterations. Data generated by FACS
or the Gene Expression / Biomarker assay were regarded as missing if
each mouse had measurements for some variables. For each continuous
variable, we specified a predictive mean matching model. All the
remaining variables were used as predictors in the imputation. To
control the quality of our imputations, we evaluated the distribution
plot of the existing data and the imputed data for all measurements.
Further, we tested for convergence \ref{fig:sup_fig1}. We assume data is
``missing completely at random'' or ``missing at random''. For both
types of missingness, multiple imputation is a suggested method to
impute missing variables \citet{van2018flexible}.

\hypertarget{questions-to-dos}{%
\paragraph{Questions / To-dos:}\label{questions-to-dos}}

\begin{enumerate}
\def\labelenumi{\arabic{enumi}.}
\tightlist
\item
  Should I log-transform the data prior to imputation?
\item
  Increasing produced data sets / iterations
\item
  sensitivity analyses using complete cases only
\end{enumerate}

\hypertarget{results}{%
\section{Results}\label{results}}

\hypertarget{discussion}{%
\section{Discussion}\label{discussion}}

\hypertarget{conclusion}{%
\section{Conclusion}\label{conclusion}}

\hypertarget{a-subsection}{%
\subsection{A subsection}\label{a-subsection}}

\hypertarget{literature-citations}{%
\section{Literature citations}\label{literature-citations}}

\hypertarget{equations}{%
\section{Equations}\label{equations}}

An equation without a label for cross-referencing:

\[
E=mc^2
\]

An inline equation: \(y=ax+b\)

An equation with a label for cross-referencing:

\begin{equation}\label{eq:eq1}
\int^{r_2}_0 F(r,\varphi){\rm d}r\,{\rm d}\varphi = 1
\end{equation}

This equation can be referenced as follows: Eq. \ref{eq:eq1}

\hypertarget{inserting-r-figures}{%
\section{Inserting R figures}\label{inserting-r-figures}}

The code below creates a figure. The code is included in the output
because \texttt{echo=TRUE}.

\begin{Shaded}
\begin{Highlighting}[]
\FunctionTok{plot}\NormalTok{(}\DecValTok{1}\SpecialCharTok{:}\DecValTok{10}\NormalTok{,}\AttributeTok{main=}\StringTok{"Some data"}\NormalTok{,}\AttributeTok{xlab=}\StringTok{"Distance (cm)"}\NormalTok{,}
     \AttributeTok{ylab=}\StringTok{"Time (hours)"}\NormalTok{)}
\end{Highlighting}
\end{Shaded}

\begin{figure}[th]
\includegraphics[width=1\linewidth]{x_files/figure-latex/fig1-1} \caption{This is the first figure.}\label{fig:fig1}
\end{figure}

You can reference this figure as follows: Fig. \ref{fig:fig1}.

\hypertarget{figures-spanning-two-columns}{%
\subsection{Figures spanning
two-columns}\label{figures-spanning-two-columns}}

Figures can span two columns be setting \texttt{fig.env="figure*"}.

\begin{figure*}[th]
\includegraphics[width=1\linewidth]{x_files/figure-latex/fig2-1} \caption{This is a wide figure.}\label{fig:fig2}
\end{figure*}

Reference to second figure: Fig. \ref{fig:fig2}

\hypertarget{tables}{%
\section{Tables}\label{tables}}

\hypertarget{generate-a-table-using-xtable}{%
\subsection{\texorpdfstring{Generate a table using
\texttt{xtable}}{Generate a table using xtable}}\label{generate-a-table-using-xtable}}

\begin{Shaded}
\begin{Highlighting}[]
\NormalTok{df }\OtherTok{=} \FunctionTok{data.frame}\NormalTok{(}\AttributeTok{ID=}\DecValTok{1}\SpecialCharTok{:}\DecValTok{3}\NormalTok{,}\AttributeTok{code=}\NormalTok{letters[}\DecValTok{1}\SpecialCharTok{:}\DecValTok{3}\NormalTok{])}

\CommentTok{\# Creates tables that follow OUP guidelines }
\CommentTok{\# using xtable}
\FunctionTok{library}\NormalTok{(xtable) }
\end{Highlighting}
\end{Shaded}

\begin{verbatim}
## Warning: package 'xtable' was built under R version 4.2.1
\end{verbatim}

\begin{Shaded}
\begin{Highlighting}[]
\FunctionTok{print}\NormalTok{(}\FunctionTok{xtable}\NormalTok{(df,}\AttributeTok{caption=}\StringTok{"This is a xtable table."}\NormalTok{,}
             \AttributeTok{label=}\StringTok{"tab:tab1"}\NormalTok{),}
      \AttributeTok{comment=}\ConstantTok{FALSE}\NormalTok{,}\AttributeTok{caption.placement=}\StringTok{"top"}\NormalTok{)}
\end{Highlighting}
\end{Shaded}

\begin{table}[ht]
\centering
\caption{This is a xtable table.} 
\label{tab:tab1}
\begin{tabular}{rrl}
  \hline
 & ID & code \\ 
  \hline
1 &   1 & a \\ 
  2 &   2 & b \\ 
  3 &   3 & c \\ 
   \hline
\end{tabular}
\end{table}

You can reference this table as follows: Table \ref{tab:tab1}.

\hypertarget{generate-a-table-using-kable}{%
\subsection{\texorpdfstring{Generate a table using
\texttt{kable}}{Generate a table using kable}}\label{generate-a-table-using-kable}}

\begin{Shaded}
\begin{Highlighting}[]
\NormalTok{df }\OtherTok{=} \FunctionTok{data.frame}\NormalTok{(}\AttributeTok{ID=}\DecValTok{1}\SpecialCharTok{:}\DecValTok{3}\NormalTok{,}\AttributeTok{code=}\NormalTok{letters[}\DecValTok{1}\SpecialCharTok{:}\DecValTok{3}\NormalTok{])}

\CommentTok{\# kable can alse be used for creating tables}
\NormalTok{knitr}\SpecialCharTok{::}\FunctionTok{kable}\NormalTok{(df,}\AttributeTok{caption=}\StringTok{"This is a kable table."}\NormalTok{,}
             \AttributeTok{booktabs=}\ConstantTok{TRUE}\NormalTok{,}\AttributeTok{label=}\StringTok{"tab2"}\NormalTok{)}
\end{Highlighting}
\end{Shaded}

\begin{table}

\caption{\label{tab:tab2}This is a kable table.}
\centering
\begin{tabular}[t]{rl}
\toprule
ID & code\\
\midrule
1 & a\\
2 & b\\
3 & c\\
\bottomrule
\end{tabular}
\end{table}

You can reference this table as follows: Table \ref{tab:tab2}.

\hypertarget{table-spanning-two-columns}{%
\subsection{Table spanning two
columns}\label{table-spanning-two-columns}}

Tables can span two columns be setting \texttt{table.envir\ =\ "table*"}
in \texttt{knitr::kable}.

\begin{Shaded}
\begin{Highlighting}[]
\NormalTok{df }\OtherTok{=} \FunctionTok{data.frame}\NormalTok{(}\AttributeTok{ID=}\DecValTok{1}\SpecialCharTok{:}\DecValTok{3}\NormalTok{,}\AttributeTok{code1=}\NormalTok{letters[}\DecValTok{1}\SpecialCharTok{:}\DecValTok{3}\NormalTok{],}
                \AttributeTok{code2=}\NormalTok{letters[}\DecValTok{4}\SpecialCharTok{:}\DecValTok{6}\NormalTok{],}
                \AttributeTok{code3=}\NormalTok{letters[}\DecValTok{7}\SpecialCharTok{:}\DecValTok{9}\NormalTok{],}
                \AttributeTok{code4=}\NormalTok{letters[}\DecValTok{10}\SpecialCharTok{:}\DecValTok{12}\NormalTok{],}
                \AttributeTok{code5=}\NormalTok{letters[}\DecValTok{13}\SpecialCharTok{:}\DecValTok{15}\NormalTok{])}

\CommentTok{\# kable can alse be used for creating tables}
\NormalTok{knitr}\SpecialCharTok{::}\FunctionTok{kable}\NormalTok{(df,}\AttributeTok{caption=}\StringTok{"This is a wide kable table."}\NormalTok{,}
             \CommentTok{\#format="latex",}
             \AttributeTok{table.envir=}\StringTok{"table*"}\NormalTok{,}
             \AttributeTok{booktabs=}\ConstantTok{TRUE}\NormalTok{,}\AttributeTok{label=}\StringTok{"tab3"}\NormalTok{)}
\end{Highlighting}
\end{Shaded}

\begin{table*}

\caption{\label{tab:tab3}This is a wide kable table.}
\centering
\begin{tabular}[t]{rlllll}
\toprule
ID & code1 & code2 & code3 & code4 & code5\\
\midrule
1 & a & d & g & j & m\\
2 & b & e & h & k & n\\
3 & c & f & i & l & o\\
\bottomrule
\end{tabular}
\end{table*}

\hypertarget{cross-referencing-sections}{%
\section{Cross-referencing sections}\label{cross-referencing-sections}}

You can cross-reference sections and subsections as follows: Section
\ref{literature-citations} and Section \ref{a-subsection}.

\textbf{\emph{Note:}} the last section in the document will be used as
the section title for the bibliography.

For more portable and flexible referencing of sections, equations,
figures and tables, use
\href{https://github.com/rstudio/bookdown}{\texttt{bookdown::pdf\_document2}}
with YAML header option \texttt{base\_format:\ rticles::oup\_article}.

\hypertarget{appendices}{%
\section*{Appendices}\label{appendices}}
\addcontentsline{toc}{section}{Appendices}

\begin{appendices}

\hypertarget{section-title-of-first-appendix}{%
\section{Section title of first
appendix}\label{section-title-of-first-appendix}}

blabla

\hypertarget{subsection-title-of-first-appendix}{%
\subsection{Subsection title of first
appendix}\label{subsection-title-of-first-appendix}}

and so on\ldots.

\end{appendices}

\hypertarget{references}{%
\section{References}\label{references}}

\hypertarget{supplementarry-material}{%
\section{Supplementarry material}\label{supplementarry-material}}

\begin{Shaded}
\begin{Highlighting}[]
\FunctionTok{stripplot}\NormalTok{(igf, }\AttributeTok{pch =} \DecValTok{20}\NormalTok{, }\AttributeTok{cex =} \FloatTok{1.2}\NormalTok{)}
\end{Highlighting}
\end{Shaded}

\begin{figure*}[th]
\includegraphics[width=1\linewidth]{x_files/figure-latex/sup_fig2-1} \caption{Stripplot of observed and imputed data}\label{fig:sup_fig2}
\end{figure*}

\begin{Shaded}
\begin{Highlighting}[]
\FunctionTok{densityplot}\NormalTok{(igf, }\AttributeTok{height =} \DecValTok{1000}\NormalTok{, }\AttributeTok{width =} \DecValTok{800}\NormalTok{)}
\end{Highlighting}
\end{Shaded}

\begin{figure*}[th]
\includegraphics[width=1\linewidth]{x_files/figure-latex/sup_fig3-1} \caption{Stripplot of observed and imputed data}\label{fig:sup_fig3}
\end{figure*}

\section{Competing interests}

There are no competing interest.

\section{Author contributions statement}

To be worked on



\bibliographystyle{abbrvnat}
\bibliography{mybibfile.bib}

%% Author bio-pics with images
\begin{biography}{%
%
}{\author{Fay Webster} jkjb}
\end{biography}


\end{document}
